% Đây chỉ là file hướng dẫn làm quen với Latex
% Khi hoàn thiện báo cáo thì các em remove đi nhé

\newpage
\section{Hướng dẫn viết báo cáo bằng \LaTeX}

Tài liệu này hướng dẫn sinh viên các thao tác cơ bản khi viết báo cáo bằng \LaTeX, bao gồm cách tạo mục, trích dẫn tài liệu, chèn hình ảnh, bảng biểu và một số lưu ý quan trọng.

% ==================================================
\subsection{Cách tạo chương}
% ==================================================

Sinh viên sử dụng các lệnh sau để tạo cấu trúc báo cáo:

\begin{itemize}
    \item \verb|\section{Tên chương}|: Tạo chương chính
    \item \verb|\subsection{Tên mục}|: Tạo mục con
    \item \verb|\subsubsection{Tên mục nhỏ}|: Tạo mục nhỏ hơn
\end{itemize}

\textbf{Lưu ý:} Không đánh số mục bằng tay, \LaTeX\ sẽ tự động quản lý.

% ==================================================
\subsection{Cách trích dẫn tài liệu}
% ==================================================

Sinh viên thêm tài liệu tham khảo vào file main.bib. Ví dụ:

\begin{verbatim}
@article{vaswani2017,
  title={Attention Is All You Need},
  author={Vaswani et al.},
  journal={NeurIPS},
  year={2017}
}
\end{verbatim}

Sau đó, trích dẫn trong nội dung bằng cách sử dụng lệnh:
\begin{verbatim} \cite{vaswani2017} \end{verbatim}

Ví dụ:
\begin{quote}
Mô hình Transformer được đề xuất trong bài báo “Attention Is All You Need” \cite{vaswani2017}.
\end{quote}

% ==================================================
\subsection{Cách chèn hình ảnh}
% ==================================================

Cú pháp chèn hình:

\begin{verbatim}
\begin{figure}[H]
    \centering
    \includegraphics[width=0.7\textwidth]{guide/example.png}
    \caption{Giao diện ứng dụng chat}
    \label{fig:chat_ui}
\end{figure}
\end{verbatim}

Ví dụ:
\begin{figure}[H]
    \centering
    \includegraphics[width=0.7\textwidth]{guide/example.png}
    \caption{Giao diện ứng dụng chat}
    \label{fig:chat_ui}
\end{figure}

% ==================================================
\subsection{Cách chèn bảng}
% ==================================================

Cú pháp chèn bảng:

\begin{verbatim}
\begin{table}[h]
\centering
\caption{So sánh các chức năng của Chat Application}
\label{tab:features}
\begin{tabular}{|l|c|l|}
\hline
\textbf{Chức năng} & \textbf{Hỗ trợ} & \textbf{Ghi chú} \\ \hline
Chat realtime      & Có              & WebSocket        \\ \hline
Chat nhóm          & Có              & --               \\ \hline
Gửi file           & Chưa            & Future work      \\ \hline
\end{tabular}
\end{table}
\end{verbatim}

Ví dụ:

\begin{table}[H]
\centering
\label{tab:features}
\begin{tabular}{|l|c|l|}
\hline
\textbf{Chức năng} & \textbf{Hỗ trợ} & \textbf{Ghi chú} \\ \hline
Chat realtime      & Có              & WebSocket        \\ \hline
Chat nhóm          & Có              & --               \\ \hline
Gửi file           & Chưa            & Future work      \\ \hline
\end{tabular}
\caption{So sánh các chức năng của Chat Application}
\end{table}

% ==================================================
\subsection{Cách chèn liên kết (URL)}
% ==================================================

Cú pháp chèn liên kết:

\begin{verbatim}
\url{https://www.example.com}

\href{https://www.example.com}{Trang web minh họa}
\end{verbatim}

Ví dụ hiển thị:

\url{https://www.example.com}

\href{https://www.example.com}{Trang web minh họa}

% ==================================================
\subsection{Cách chèn công thức toán học}
% ==================================================

\subsubsection{Công thức trong dòng:}

\begin{verbatim}
$ E = mc^2 $
\end{verbatim}

Ví dụ: $ E = mc^2 $

\subsubsection{Công thức hiển thị riêng dòng:}

\begin{verbatim}
\[
a^2 + b^2 = c^2
\]
\end{verbatim}

\[
a^2 + b^2 = c^2
\]

\subsubsection{Công thức có đánh số:}

\begin{verbatim}
\begin{equation}
a^2 + b^2 = c^2
\end{equation}
\end{verbatim}

\begin{equation}
\label{eq:pythagoras}
a^2 + b^2 = c^2
\end{equation}

