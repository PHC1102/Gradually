\section{Tổng quan hệ thống}

\subsection{Mô tả chức năng}

Ứng dụng Gradually cung cấp các nhóm chức năng chính sau:

\subsubsection{Quản lý người dùng và xác thực}
\begin{itemize}
    \item Đăng ký tài khoản với xác thực email
    \item Đăng nhập/Đăng xuất
    \item Thiết lập hồ sơ cá nhân (Profile Setup)
    \item Quản lý phiên làm việc
\end{itemize}

\subsubsection{Quản lý tổ chức (Organization)}
\begin{itemize}
    \item Tạo tổ chức mới
    \item Mời thành viên qua email
    \item Phân quyền: Owner (chủ sở hữu), Admin (quản trị), Member (thành viên)
    \item Chuyển đổi giữa các tổ chức
    \item Rời khỏi hoặc xóa tổ chức
\end{itemize}

\subsubsection{Quản lý dự án (Project)}
\begin{itemize}
    \item Tạo dự án trong tổ chức
    \item Thêm thành viên vào dự án
    \item Phân quyền dự án: Admin, Contributor (cộng tác viên), Viewer (người xem)
    \item Rời khỏi hoặc xóa dự án
\end{itemize}

\subsubsection{Quản lý công việc (Task Management)}
\begin{itemize}
    \item Tạo, sửa, xóa task với các thuộc tính: tiêu đề, mô tả, deadline, priority, status, assignee
    \item Quản lý subtask thủ công
    \item \textbf{Sinh subtask tự động bằng AI} - tính năng cốt lõi của ứng dụng
    \item Đánh dấu hoàn thành task/subtask
    \item Sắp xếp theo thời gian, deadline, priority
\end{itemize}

\subsubsection{Đa dạng chế độ xem}
\begin{itemize}
    \item \textbf{List View:} Hiển thị danh sách task dạng thẻ
    \item \textbf{Board View:} Hiển thị dạng Kanban với các cột trạng thái (To Do, In Progress, Done)
    \item \textbf{Calendar View:} Hiển thị task theo lịch, dễ dàng theo dõi deadline
    \item \textbf{Completed View:} Xem các task đã hoàn thành
    \item \textbf{Analysis View:} Thống kê và phân tích tiến độ công việc
\end{itemize}

\subsubsection{Hệ thống thông báo}
\begin{itemize}
    \item Cảnh báo task sắp đến hạn
    \item Thông báo task quá hạn (overdue)
    \item Tự động dọn dẹp thông báo đã xử lý
\end{itemize}

\subsection{Biểu đồ Use Case}

% TODO: Chèn ảnh Use Case Diagram
% Tạo file: method/usecase.png
\begin{figure}[H]
    \centering
    % \includegraphics[width=0.9\textwidth]{method/usecase.png}
    \fbox{\parbox{0.8\textwidth}{\centering\vspace{3cm}\textbf{[CHÈN ẢNH: Use Case Diagram]}\\\small Vẽ bằng draw.io hoặc PlantUML\\Lưu tại: method/usecase.png\vspace{3cm}}}
    \caption{Biểu đồ Use Case của hệ thống Gradually}
    \label{fig:usecase}
\end{figure}

Biểu đồ Use Case (Hình \ref{fig:usecase}) thể hiện các actor chính và các chức năng tương ứng:
\begin{itemize}
    \item \textbf{Guest:} Đăng ký, Đăng nhập
    \item \textbf{Member:} Xem task, Cập nhật trạng thái task
    \item \textbf{Contributor:} Tất cả quyền của Member + Chỉnh sửa task
    \item \textbf{Admin:} Tất cả quyền của Contributor + Tạo/Xóa task, Quản lý thành viên project
    \item \textbf{Owner:} Toàn quyền quản lý tổ chức và các dự án
\end{itemize}

%==================================================
\section{Thiết kế hệ thống}
%==================================================

\subsection{Kiến trúc tổng thể}

Ứng dụng Gradually được xây dựng theo kiến trúc \textbf{Serverless} với Firebase làm backend, giúp giảm thiểu chi phí vận hành và dễ dàng mở rộng.

% TODO: Chèn ảnh Architecture Diagram
\begin{figure}[H]
    \centering
    % \includegraphics[width=0.85\textwidth]{method/architecture.png}
    \fbox{\parbox{0.8\textwidth}{\centering\vspace{4cm}\textbf{[CHÈN ẢNH: System Architecture]}\\\small Vẽ sơ đồ kiến trúc hệ thống\\Lưu tại: method/architecture.png\vspace{4cm}}}
    \caption{Kiến trúc tổng thể của hệ thống Gradually}
    \label{fig:architecture}
\end{figure}

\subsubsection{Các thành phần chính}

\begin{table}[H]
\centering
\caption{Các thành phần trong kiến trúc hệ thống}
\label{tab:components}
\begin{tabular}{|l|l|p{7cm}|}
\hline
\textbf{Thành phần} & \textbf{Công nghệ} & \textbf{Mô tả} \\ \hline
Frontend & React 19 + TypeScript & Single Page Application với Vite build tool \\ \hline
State Management & Zustand & Quản lý state nhẹ, selector + shallow compare tối ưu re-render, persist theo user \\ \hline
Authentication & Firebase Auth & Xác thực email/password với email verification \\ \hline
Database & Cloud Firestore & NoSQL database với real-time sync \\ \hline
AI Service & External API & Tích hợp AI để sinh subtask tự động \\ \hline
\end{tabular}
\end{table}

\subsubsection{Frontend architecture \& state flow}
\begin{itemize}
    \item Tổ chức \textbf{feature-first} (components/pages/services/store) để giảm coupling, thuận tiện mở rộng.
    \item \textbf{Zustand} dùng selector + shallow compare để tránh re-render, persist state gắn với user (setCurrentUserId).
    \item \textbf{AuthContext} cung cấp currentUser/isEmailVerified/profileLoading; App.tsx điều phối view và guard truy cập.
    \item \textbf{UI/UX}: Sidebar nhóm chức năng kiểu Facebook, nhiều chế độ xem (List/Board/Calendar/Analysis) giảm tải nhận thức.
\end{itemize}

% TODO: Chèn ảnh Frontend state flow
\begin{figure}[H]
    \centering
    % \includegraphics[width=0.85\textwidth]{method/state_flow.png}
    \fbox{\parbox{0.8\textwidth}{\centering\vspace{4cm}\textbf{[CHÈN ẢNH: Frontend State Flow]}\\\small Luồng state giữa AuthContext, Zustand stores và views\\Lưu tại: method/state_flow.png\vspace{4cm}}}
    \caption{Luồng state và dữ liệu trên frontend}
    \label{fig:state_flow}
\end{figure}

\subsubsection{Authentication \& phân quyền}
\begin{itemize}
    \item Đăng nhập Firebase Auth, bắt buộc \textbf{email verification}; sau login load organizations và validate selectedOrgId.
    \item Subscribe members để lấy role org (owner/admin/member); subscribe projectMembers để lấy role project (admin/contributor/viewer).
    \item Guard UI tại App.tsx: các return sớm (Loading, Login, Verify email, Select Org/Project) đảm bảo hook order và chặn truy cập trái phép.
    \item Quyền hành động: chỉ admin/owner được tạo task; admin/contributor chỉnh sửa; viewer chỉ xem.
\end{itemize}

% TODO: Chèn Sequence đăng nhập và lấy role
\begin{figure}[H]
    \centering
    % \includegraphics[width=0.85\textwidth]{method/sequence_auth.png}
    \fbox{\parbox{0.8\textwidth}{\centering\vspace{4cm}\textbf{[CHÈN ẢNH: Auth \& Role Sequence]}\\\small Luồng: Login -> Load Orgs -> Validate Org -> Load Projects -> Roles\\Lưu tại: method/sequence_auth.png\vspace{4cm}}}
    \caption{Luồng xác thực và phân quyền}
    \label{fig:sequence_auth}
\end{figure}

\subsubsection{Real-time sync \& cleanup}
\begin{itemize}
    \item Firestore listeners: \texttt{subscribeToOrganizationMembers}, \texttt{subscribeToProjectMembers}, \texttt{subscribeToProjectTasks}.
    \item Cleanup: return unsubscribe trong useEffect; khi đổi org/project hoặc logout thì clear store, tránh rò rỉ listener.
    \item Khi selectedOrgId/selectedProjectId null: reset tasks, completedTasks, members, roles về trạng thái rỗng.
    \item TaskForm chỉ render khi appController.shouldShowForm hoặc đang edit task/subtask, tránh mount không cần thiết.
\end{itemize}

% TODO: Chèn Sequence real-time tasks
\begin{figure}[H]
    \centering
    % \includegraphics[width=0.85\textwidth]{method/sequence_realtime.png}
    \fbox{\parbox{0.8\textwidth}{\centering\vspace{4cm}\textbf{[CHÈN ẢNH: Realtime Task Sync Sequence]}\\\small Luồng: Firestore -> listener -> taskManager -> setTasks state\\Lưu tại: method/sequence_realtime.png\vspace{4cm}}}
    \caption{Luồng đồng bộ thời gian thực cho Task}
    \label{fig:sequence_realtime}
\end{figure}

\subsection{Thiết kế cơ sở dữ liệu}

Hệ thống sử dụng Cloud Firestore với cấu trúc document-based. Dưới đây là thiết kế schema chính:

% TODO: Chèn ảnh Database Schema
\begin{figure}[H]
    \centering
    % \includegraphics[width=0.9\textwidth]{method/database_schema.png}
    \fbox{\parbox{0.85\textwidth}{\centering\vspace{4cm}\textbf{[CHÈN ẢNH: Database Schema]}\\\small Vẽ sơ đồ quan hệ các collection\\Lưu tại: method/database\_schema.png\vspace{4cm}}}
    \caption{Sơ đồ cơ sở dữ liệu Firestore}
    \label{fig:database}
\end{figure}

\subsubsection{Cấu trúc Collections}

\textbf{1. Collection: users}
\begin{verbatim}
users/{userId}
├── userId: string
├── email: string
├── emailLower: string (for case-insensitive search)
├── displayName: string
├── avatarUrl: string
└── joinedOrganizations: array<string>
\end{verbatim}

\textbf{2. Collection: organizations}
\begin{verbatim}
organizations/{orgId}
├── name: string
├── ownerId: string
├── ownerEmail: string
├── createdAt: timestamp
├── memberCount: number
└── members/{userId}  (subcollection)
    ├── role: "owner" | "admin" | "member"
    ├── email: string
    ├── displayName: string
    └── joinedAt: timestamp
\end{verbatim}

\textbf{3. Collection: projects (subcollection of organizations)}
\begin{verbatim}
organizations/{orgId}/projects/{projectId}
├── name: string
├── description: string
├── status: "active" | "archived"
├── createdAt: timestamp
├── members/{userId}  (subcollection)
│   ├── role: "admin" | "contributor" | "viewer"
│   └── addedAt: timestamp
└── tasks/{taskId}  (subcollection)
    ├── title: string
    ├── description: string
    ├── deadline: string (ISO datetime)
    ├── status: "todo" | "inProgress" | "done"
    ├── priority: "low" | "medium" | "high" | "critical"
    ├── assigneeId: string
    ├── done: boolean
    ├── createdAt: number
    └── subtasks: array<Subtask>
\end{verbatim}

\textbf{4. Collection: organizationInvitations}
\begin{verbatim}
organizationInvitations/{invitationId}
├── orgId: string
├── invitedEmail: string
├── invitedUserId: string (optional)
├── invitedBy: string
├── status: "pending" | "accepted" | "declined"
└── createdAt: timestamp
\end{verbatim}

\subsection{Thiết kế giao diện người dùng}

Giao diện ứng dụng được thiết kế theo nguyên tắc \textbf{Dark Mode First} với bảng màu chủ đạo:
\begin{itemize}
    \item Background: \#1a1a2e (Dark Navy)
    \item Primary: \#5FF281 (Neon Green)
    \item Secondary: \#16213e (Dark Blue)
    \item Accent colors cho priority: Green (Low), Yellow (Medium), Red (High), Dark Red (Critical)
\end{itemize}

% TODO: Chèn ảnh UI Design / Mockup
\begin{figure}[H]
    \centering
    % \includegraphics[width=0.9\textwidth]{method/ui_overview.png}
    \fbox{\parbox{0.85\textwidth}{\centering\vspace{4cm}\textbf{[CHÈN ẢNH: UI Overview]}\\\small Screenshot tổng quan giao diện\\Lưu tại: method/ui\_overview.png\vspace{4cm}}}
    \caption{Tổng quan giao diện ứng dụng Gradually}
    \label{fig:ui_overview}
\end{figure}

\subsection{Thiết kế tính năng AI Subtask Generation}

Đây là tính năng cốt lõi của ứng dụng, giúp người dùng phân rã task lớn thành các subtask nhỏ hơn một cách tự động.

\subsubsection{Luồng hoạt động}

\begin{enumerate}
    \item Người dùng tạo task với tiêu đề và deadline
    \item Nhấn nút ``AI Generate'' trong form tạo task
    \item Frontend gửi request đến AI Service với thông tin task
    \item AI phân tích và trả về danh sách subtask được đề xuất
    \item Các subtask được thêm vào task với deadline được tính toán hợp lý
\end{enumerate}

\subsubsection{Thuật toán phân bổ thời gian}

AI Service không chỉ sinh ra danh sách subtask mà còn tính toán deadline cho từng subtask dựa trên:
\begin{itemize}
    \item Tổng thời gian còn lại đến deadline của task cha
    \item Độ phức tạp ước tính của từng subtask
    \item Phân bổ đều thời gian để tránh dồn việc (anti-cramming)
\end{itemize}

% TODO: Chèn Sequence AI
\begin{figure}[H]
    \centering
    % \includegraphics[width=0.8\textwidth]{method/ai_sequence.png}
    \fbox{\parbox{0.75\textwidth}{\centering\vspace{4cm}\textbf{[CHÈN ẢNH: AI Subtask Generation Sequence]}\\\small Luồng: User -> Frontend -> AI API -> Subtasks -> Firestore\\Lưu tại: method/ai_sequence.png\vspace{4cm}}}
    \caption{Trình tự sinh subtask bằng AI}
    \label{fig:ai_sequence}
\end{figure}

\subsubsection{Heuristic chống cramming}
\begin{itemize}
    \item Nếu còn $T$ giờ đến deadline và có $n$ subtask: phân bổ cơ sở $T/n$, dành \textbf{20--30\% buffer} cuối cho rủi ro.
    \item Subtask phức tạp hơn sẽ được đẩy lên sớm (front-loading) để giảm nguy cơ dồn việc.
    \item Nếu người dùng sửa deadline subtask thủ công, heuristic chỉ tính lại cho phần chưa khóa.
\end{itemize}

% TODO: Chèn biểu đồ Gantt phân bổ subtask
\begin{figure}[H]
    \centering
    % \includegraphics[width=0.85\textwidth]{method/anti_cramming_gantt.png}
    \fbox{\parbox{0.8\textwidth}{\centering\vspace{4cm}\textbf{[CHÈN ẢNH: Anti-cramming Gantt]}\\\small Minh họa phân bổ subtask tránh dồn việc\\Lưu tại: method/anti_cramming_gantt.png\vspace{4cm}}}
    \caption{Gantt minh họa phân bổ subtask tránh cramming}
    \label{fig:anti_cramming}
\end{figure}

\subsubsection{Bảo mật \& giới hạn AI}
\begin{itemize}
    \item API key AI đặt phía server/proxy, không để lộ trên client.
    \item Rate limit theo user/ngày; fallback sang tạo subtask thủ công khi vượt hạn mức.
    \item Retry tối đa 2 lần; nếu vẫn lỗi, hiển thị thông báo và giữ nguyên dữ liệu người dùng đã nhập.
\end{itemize}

%==================================================
\section{Triển khai hệ thống}
%==================================================

\subsection{Cấu trúc source code}

Dự án được tổ chức theo mô hình feature-based với cấu trúc thư mục rõ ràng:

\begin{verbatim}
src/
├── components/          # React components
│   ├── org/            # Organization-related components
│   ├── project/        # Project-related components
│   └── ...
├── contexts/           # React Context (AuthContext)
├── pages/              # Page components
├── services/           # Business logic & API calls
│   ├── authService.ts
│   ├── organizationService.ts
│   ├── projectService.ts
│   ├── firebaseTaskService.ts
│   ├── aiService.ts
│   └── notificationService.ts
├── store/              # Zustand state stores
│   ├── organizationStore.ts
│   └── projectStore.ts
├── utils/              # Utility functions
├── types.ts            # TypeScript type definitions
├── App.tsx             # Main application component
└── main.tsx            # Entry point
\end{verbatim}

\subsection{Các module chính}

\subsubsection{Authentication Module}

Xử lý đăng ký, đăng nhập, xác thực email và quản lý phiên:

\begin{itemize}
    \item \texttt{AuthContext.tsx}: React Context cung cấp thông tin user cho toàn ứng dụng
    \item \texttt{authService.ts}: Các hàm tương tác với Firebase Auth
    \item \texttt{Login.tsx}: Component giao diện đăng nhập/đăng ký
\end{itemize}

\subsubsection{Organization \& Project Module}

Quản lý cấu trúc tổ chức với phân quyền:

\begin{itemize}
    \item Real-time subscription để đồng bộ danh sách members
    \item Hệ thống invitation với email
    \item Validation quyền trước mỗi thao tác
\end{itemize}

\subsubsection{Task Management Module}

Xử lý CRUD cho task và subtask với các tính năng:

\begin{itemize}
    \item Firestore real-time listener cho task updates
    \item Sorting theo multiple criteria (time, deadline, priority)
    \item Drag-and-drop trong Board View
\end{itemize}

\subsubsection{AI Service Module}

Tích hợp AI để sinh subtask:

\begin{itemize}
    \item Gửi prompt chứa thông tin task đến AI API
    \item Parse response và tạo subtask objects
    \item Tính toán deadline phù hợp cho từng subtask
\end{itemize}

\subsection{Kết quả triển khai}

\subsubsection{Giao diện List View}

% TODO: Chèn screenshot List View
\begin{figure}[H]
    \centering
    % \includegraphics[width=0.9\textwidth]{method/screenshot_list.png}
    \fbox{\parbox{0.85\textwidth}{\centering\vspace{4cm}\textbf{[CHÈN ẢNH: List View Screenshot]}\\\small Screenshot màn hình List View\\Lưu tại: method/screenshot\_list.png\vspace{4cm}}}
    \caption{Giao diện List View - hiển thị danh sách task dạng thẻ}
    \label{fig:list_view}
\end{figure}

\subsubsection{Giao diện Board View (Kanban)}

% TODO: Chèn screenshot Board View
\begin{figure}[H]
    \centering
    % \includegraphics[width=0.9\textwidth]{method/screenshot_board.png}
    \fbox{\parbox{0.85\textwidth}{\centering\vspace{4cm}\textbf{[CHÈN ẢNH: Board View Screenshot]}\\\small Screenshot màn hình Board View\\Lưu tại: method/screenshot\_board.png\vspace{4cm}}}
    \caption{Giao diện Board View - Kanban với các cột trạng thái}
    \label{fig:board_view}
\end{figure}

\subsubsection{Giao diện Calendar View}

% TODO: Chèn screenshot Calendar View
\begin{figure}[H]
    \centering
    % \includegraphics[width=0.9\textwidth]{method/screenshot_calendar.png}
    \fbox{\parbox{0.85\textwidth}{\centering\vspace{4cm}\textbf{[CHÈN ẢNH: Calendar View Screenshot]}\\\small Screenshot màn hình Calendar View\\Lưu tại: method/screenshot\_calendar.png\vspace{4cm}}}
    \caption{Giao diện Calendar View - xem task theo lịch}
    \label{fig:calendar_view}
\end{figure}

\subsubsection{Form tạo Task với AI Generate}

% TODO: Chèn screenshot Task Form với nút AI
\begin{figure}[H]
    \centering
    % \includegraphics[width=0.7\textwidth]{method/screenshot_ai_generate.png}
    \fbox{\parbox{0.65\textwidth}{\centering\vspace{4cm}\textbf{[CHÈN ẢNH: AI Generate Feature]}\\\small Screenshot form tạo task với nút AI Generate\\Lưu tại: method/screenshot\_ai\_generate.png\vspace{4cm}}}
    \caption{Form tạo Task với tính năng AI sinh subtask tự động}
    \label{fig:ai_generate}
\end{figure}

\subsubsection{Sidebar Navigation}

% TODO: Chèn screenshot Sidebar
\begin{figure}[H]
    \centering
    % \includegraphics[width=0.4\textwidth]{method/screenshot_sidebar.png}
    \fbox{\parbox{0.35\textwidth}{\centering\vspace{5cm}\textbf{[CHÈN ẢNH: Sidebar]}\\\small Screenshot sidebar navigation\\Lưu tại: method/screenshot\_sidebar.png\vspace{5cm}}}
    \caption{Sidebar điều hướng theo nhóm chức năng}
    \label{fig:sidebar}
\end{figure}

\subsubsection{Profile Menu}

% TODO: Chèn screenshot Account Menu
\begin{figure}[H]
    \centering
    % \includegraphics[width=0.4\textwidth]{method/screenshot_profile.png}
    \fbox{\parbox{0.35\textwidth}{\centering\vspace{4cm}\textbf{[CHÈN ẢNH: Profile Menu]}\\\small Screenshot menu profile\\Lưu tại: method/screenshot\_profile.png\vspace{4cm}}}
    \caption{Menu tài khoản với thông tin user và chuyển đổi tổ chức/dự án}
    \label{fig:profile_menu}
\end{figure}

\subsubsection{Analysis View}

% TODO: Chèn screenshot Analysis View
\begin{figure}[H]
    \centering
    % \includegraphics[width=0.9\textwidth]{method/screenshot_analysis.png}
    \fbox{\parbox{0.85\textwidth}{\centering\vspace{4cm}\textbf{[CHÈN ẢNH: Analysis View Screenshot]}\\\small Screenshot màn hình Analysis View\\Lưu tại: method/screenshot\_analysis.png\vspace{4cm}}}
    \caption{Giao diện Analysis View - thống kê tiến độ}
    \label{fig:analysis_view}
\end{figure}

\subsection{Bảng tổng hợp chức năng}

\begin{table}[H]
\centering
\caption{Tổng hợp các chức năng đã triển khai}
\label{tab:features_summary}
\begin{tabular}{|l|c|p{6cm}|}
\hline
\textbf{Chức năng} & \textbf{Trạng thái} & \textbf{Ghi chú} \\ \hline
Đăng ký/Đăng nhập & Hoàn thành & Email verification bắt buộc \\ \hline
Quản lý Organization & Hoàn thành & CRUD + phân quyền \\ \hline
Quản lý Project & Hoàn thành & CRUD + phân quyền \\ \hline
Quản lý Task & Hoàn thành & CRUD + assignee + priority \\ \hline
Quản lý Subtask & Hoàn thành & CRUD thủ công \\ \hline
AI sinh Subtask & Hoàn thành & Tích hợp AI API \\ \hline
List View & Hoàn thành & Sorting đa tiêu chí \\ \hline
Board View (Kanban) & Hoàn thành & Drag-and-drop \\ \hline
Calendar View & Hoàn thành & Hiển thị theo tháng \\ \hline
Analysis View & Hoàn thành & Thống kê tiến độ \\ \hline
Thông báo deadline & Hoàn thành & Overdue alerts \\ \hline
Real-time sync & Hoàn thành & Firestore listeners \\ \hline
\end{tabular}
\end{table}
